\documentclass[a4paper, 12 pt]{article}
% perl latexindent.pl C:\Users\Dmitriy\Desktop\programming\LATEX\index.tex
\usepackage[english, russian]{babel}
\usepackage[T1,T2A]{fontenc}
\usepackage[utf8]{inputenc}
\usepackage{indentfirst}

\usepackage{geometry} % настройки страницы
\geometry{top=25mm}
\geometry{bottom=30mm}
\geometry{left=20mm}
\geometry{right=20mm}
\linespread{1}
\setlength{\parindent}{50pt}
\setlength{\parskip}{12pt}

\usepackage{titleps} % колонтитулы
\newpagestyle{main}{
	% \setheadrule{0.4pt}
	% \sethead{лево}{центр}{право}
	\setfootrule{0.4pt}
	\setfoot{\thepage}{}{}
}
\pagestyle{main}

\usepackage{graphicx}
\graphicspath{{images/}}
\usepackage{svg}
\svgpath{{images/}}
\usepackage{float}

\usepackage{array}
\usepackage{arraycols}

\usepackage{amsmath, amsfonts, amssymb, amsthm, mathtools}
% TikZ pgfplots luaLaTeX  PlantUML 
\usepackage{tikz}
\usetikzlibrary{calc}
\usetikzlibrary{patterns}
\usetikzlibrary{intersections}
\usepackage{pgfplots} % графики
\pgfplotsset{compat=1.9} 

\begin{document}

\section*{Греческий алфавит}
\begin{flushleft}
	\verb|\alpha|      --- $\alpha$      $\alpha$      \\
	\verb|\beta|       --- $\beta$       $\beta$       \\
	\verb|\gamma|      --- $\gamma$      $\Gamma$      \\
	\verb|\delta|      --- $\delta$      $\Delta$      \\
	\verb|\epsilon|    --- $\epsilon$    $\epsilon$    \\
	\verb|\varepsilon| --- $\varepsilon$ $\varepsilon$ \\
	\verb|\zeta|       --- $\zeta$       $\zeta$       \\
	\verb|\eta|        --- $\eta$        $\eta$        \\
	\verb|\theta|      --- $\theta$      $\Theta$      \\
	\verb|\vartheta|   --- $\vartheta$   $\vartheta$   \\
	\verb|\iota|       --- $\iota$       $\iota$       \\
	\verb|\kappa|      --- $\kappa$      $\kappa$      \\
	\verb|\lambda|     --- $\lambda$     $\Lambda$     \\
	\verb|\mu|         --- $\mu$         $\mu$         \\
	\verb|\nu|         --- $\nu$         $\nu$         \\
	\verb|\xi|         --- $\xi$         $\Xi$         \\
	\verb|\pi|         --- $\pi$         $\Pi$         \\
	\verb|\varpi|      --- $\varpi$      $\varpi$      \\
	\verb|\rho|        --- $\rho$        $\rho$        \\
	\verb|\varrho|     --- $\varrho$     $\varrho$     \\
	\verb|\sigma|      --- $\sigma$      $\Sigma$      \\
	\verb|\varsigma|   --- $\varsigma$   $\varsigma$   \\
	\verb|\tau|        --- $\tau$        $\tau$        \\
	\verb|\upsilon|    --- $\upsilon$    $\Upsilon$    \\
	\verb|\psi|        --- $\psi$        $\Psi$        \\
	\verb|\omega|      --- $\omega$      $\Omega$      \\
\end{flushleft}

\part{Введение в Математику}
\section{Множества чисел}
\begin{flushleft}
	\verb|\mathbb{P}| $\mathbb{P}$ --- Простые числа        --- $2,3,5,7,11,13 \cdots$ делятся только на себя и на 1.\\
	\verb|\mathbb{N}| $\mathbb{N}$ --- Натуральные числа    --- $1, 2, 3$                \\
	\verb|\mathbb{Z}| $\mathbb{Z}$ --- Целые                --- $-3, 0, 3$               \\
	\verb|\mathbb{Q}| $\mathbb{Q}$ --- Рациональные числа   --- $-3.25, \frac{3}{5} $    \\
	\verb|\mathbb{R}| $\mathbb{R}$ --- Вещественные (Действительные) числа   --- $\pi$ все числа на числовой прямой.\\
	\verb|\mathbb{I}| $\mathbb{I}$ --- Иррациональные числа --- $\sqrt[2]{2}$						 \\
	\verb|\mathbb{W}| $\mathbb{W}$ --- Целые числа          ---                          \\
	\verb|\mathbb{C}| $\mathbb{C}$ --- Комплексные числа    ---             \\

	\verb|\mathbb{H}| $\mathbb{H}$ --- quaternions using     \\
	\verb|\mathbb{O}| $\mathbb{O}$ --- octonions using       \\
	\verb|\mathbb{S}| $\mathbb{S}$ --- sedenions using       \\
\end{flushleft}

\begin{figure}[H]
	\centering
	\includesvg[width=0.75\textwidth]{sets_of_numbers}%
	\caption{Множества чисел (Диаграмма Эйлера)}\label{fig:sets_of_numbers}
\end{figure}

\subsection{Операции над числами}
\begin{flushleft}
	\begin{itemize}
		\item Закон перестановочный (комутативный) --- Перемена мест слагаемых не влияет на результат суммы
		      \[a+b=b+a\]
		\item Перестановочный (комутативный) закон произведения
		      \[a \cdot b = b \cdot a\]
		\item Закон сочетатательный (ассоциативный)
		      \[(a+b)+c=a+(b+c)=a+b+c\]
		\item Cочетатательный (ассоциативный) закон произведения
		      \[(a \cdot b) \cdot c=a \cdot (b \cdot c)=a \cdot b \cdot c\]
		\item Дистрибутивный закон
		      \[a\cdot(b+c)=a\cdot b+a\cdot c\]

	\end{itemize}
\end{flushleft}

\section{Степени и их свойства}
\begin{flushleft}
	\[a^{1}=a\]
	\[a^{0}=1\]
	\[a^{n}a^{m}=a^{n+m}\]
	\[a^{n^{m}}=a^{m^{n}}=a^{n\cdot m}\]
	\[a^{n-1}=\frac{1}{a^{n}}\]
	\[\sqrt[n]{a}=a^{\frac{1}{n}} \]
	\[n \in \mathbb{N}; -b^{2n}=b^{2n} \]
	\[-b^{2n+1}=-b^{2n+1}\]
\end{flushleft}

\section{Формулы сокращенного умножения}
\begin{flushleft}
	\[{(a+b)}^{2}=a^{2}+2ab+b^{2}\]
	\[{(a-b)}^{2}=a^{2}-2ab+b^{2}\]
	\[{(a+b)}^{3}=a^{3}+3a^{2}b+3ab^{2}+b^{3}\]
	\[{(a-b)}^{3}=a^{3}-3a^{2}b+3ab^{2}-b^{3}\]
	\[a^2-b^2=(a+b)(a-b)\]
	\[a^3+b^3=(a+b)(a^{2}-2ab+b^{2})\]
	\[a^3-b^3=(a-b)(a^{2}+2ab+b^{2})\]
\end{flushleft}

\subsection{Геометрическая интерпретация}
\begin{flushleft}
	${(a+b)}^{2}=a^{2}+2ab+b^{2}$
	\\

	\begin{tikzpicture}
		\draw[black, thick] (0,0) rectangle (3,3);
		\draw[black, thick] (0,0) rectangle (2,2);
		\draw[black, thick] (2,2) -- (3,2);
		\draw[black, thick] (2,2) -- (2,3);

		\node at (-0.2,1) {a};
		\node at (1,-0.2) {a};

		\node at (-0.2,2.5) {b};
		\node at (2.5,-0.2) {b};

		\node at (1,1) {$a^2$};
		\node at (2.5,2.5) {$b^2$};
		\node at (1,2.5) {$ab$};
		\node at (2.5,1) {$ab$};

	\end{tikzpicture}
	\\

	${(a-b)}^{2}=a^{2}-2ab+b^{2}$
	\\
	\begin{tikzpicture}
		\draw[black, thick] (0,0) rectangle (3,3);
		\draw[black, thick] (0,0) rectangle (2,2);
		\draw[black, thick] (2,2) -- (3,2);
		\draw[black, thick] (2,2) -- (2,3);
		\draw[pattern=north east lines, pattern color=black] (0,0) rectangle (2,2);


		\node at (3.2,1.5) {a};
		\node at (1.5,3.2) {a};

		\node at (-0.2,2.5) {b};
		\node at (2.5,-0.2) {b};

		\node at (-0.3,1) {a-b};
		\node at (1,-0.2) {a-b};
	\end{tikzpicture}
	\\
	$a^2-b^2=(a+b)(a-b)$
	\\
	\begin{tikzpicture}
		\draw[black, thick] (0,0) rectangle (3,3);
		\draw[black, thick] (0,0) rectangle (2,2);
		\draw[pattern=north east lines, pattern color=black] (0,2) -- (0,3) -- (3,3) -- (3,0) -- (2,0) -- (2,2);
		\draw[black, thick] (2,2) -- (3,2);

		\node at (1.5,3.2) {a};
		\node at (-0.2,1) {b};

		\node at (-0.3,2.5) {a-b};
		\node at (2.5,-0.2) {a-b};

		\node at (1.5,2.5) {$S_1$};
		\node at (2.5,1) {$S_2$};
	\end{tikzpicture}
	\\
	$a^2-b^2=S_1+S_2$\\
	$S_1=a(a-b)$\\
	$S_2=b(a-b)$\\
	$S_1+S_2=a(a-b)+b(a-b)=(a+b)(a-b)$
\end{flushleft}

\section{НОК и НОД}
\begin{flushleft}
	$A,B \in \mathbb{Z}$\\
	$C = \frac{A}{B} \in \mathbb{Z}$\\
	A --- Кратное\\
	B --- Делитель\\
	$A = C \cdot B, C \in \mathbb{Z}$\\
	$\frac{A}{B}=\frac{ \not a \cdot \not b \cdot \not c \cdot d}{\not a\cdot \not b \cdot \not c}=d$

	$\mathbb{P}$ --- Простые числа        --- $2,3,5,7,11,13 \cdots$ делятся только на себя и на 1.\\
\end{flushleft}

\section{Пропорции и их свойства}
\begin{flushleft}
	\[a,b,c,d \in \mathbb R\]
	\[a,b,c,d \neq 0\]
	\[\frac{a}{b}=\frac{c}{d} -> ab=cd -> \frac{c}{a}=\frac{d}{b}\]
\end{flushleft}

\section{Обыкновенные дроби и операции над ними}
\begin{flushleft}
\end{flushleft}

\section{Системы счисления}
\begin{flushleft}
	Разряды\\
	$ 1\textsuperscript{4}2\textsuperscript{3}3\textsuperscript{2}4\textsuperscript{1}5\textsuperscript{0}.6\textsuperscript{-1}7\textsuperscript{-2}$\\
	$C_b=\sum_{i}b^{i}t_{i}$\\
	$4_{10}=1\textsuperscript{t2}0\textsuperscript{t1}0\textsuperscript{t0}_2=\sum_{i}2^{i}t_i=0 \cdot 2\textsuperscript{0}+0\cdot 2\textsuperscript{1}+1 \cdot 2\textsuperscript{2}$\\
\end{flushleft}

\section*{}
\begin{flushleft}
\end{flushleft}

\part{Дискретная математика}
\section{Комбинаторика}
\subsection{Перестановки и размещения}
\subsection{Сочетания}
\subsection{Размещения с повторениями}
\subsection{Перестановки с повторениями}
\begin{flushleft}
\end{flushleft}


\end{document}
